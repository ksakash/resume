\begin{document}

\lastupdated
\namesection{AKASH}{KUMAR SINGH}{ 
\href{mailto:ksakash@iitk.ac.in}{ksakash@iitk.ac.in} | +91-7318019013 | \href{github.com/ksakash}{Github:ksakash}
}

\section{Experience}
\runsubsection{Autonomous Underwater Vehicle}
\descript{| underwater robotics team, iitk }
\location{February 2017 – Present | Prof. Sachin Y. Shinde}
\begin{tightemize}
\item Developed a \href{https://github.com/ksakash/auv2017-1/blob/IP/task_handler_layer/ip.md}{image processing pipeline}, which can enhance raw underwater images coming from the live video stream through a camera, denoise it and threshold it, to get essential information about objects present before the bot. 
\item Added a feature to a ROS package which enables to save its node parameters during runtime and save it in a yaml file with the help of \href{http://wiki.ros.org/dynamic_reconfigure}{dynamic\_reconfigure} package provided by ROS.
\item Made \href{https://github.com/ksakash/IP_SAUVC/}{Vision Processing ROS package} which takes images from a continuous video stream coming from the camera and perform image processing on it to get essential information about objects present before the bot with features like dynamic reconfiguration of node parameters, using OpenCV library and implemented in c++, in order to perform particular tasks in \href{https://sauvc.org/}{SAUVC 2018}.
\item Worked on Feature Extraction and Matching using various algorithms like SURF, SHIFT, etc. to recognize a known object in a particular scene.
\item Working on developing image enhancement techniques for underwater images.
\item Working on object recognition and object classification using machine learning to classify between the targets underwater.
\end{tightemize}

\runsubsection{TIC-TAC-TOE}
\descript{| REINFORCEMENT LEARNING}
\location{Nov 2017 - Present | Prof. Nisheeth Srivastava, CSE, IITK}
\begin{tightemize}
\item Developing a game agent to learn the game of TIC-TAC-TOE, of varied sizes, using reinforcement learning to find out the trade-off in the performance between using GREEDY policy over EXPLORATORY policy with the increasing grid size.
\end{tightemize}
\runsubsection{CLUB AUTOMATION}
\descript{| Winter Camp }
\location{December 2016| Robotics Club, IITK}
\begin{tightemize}
\item Developed a motion detector automate the light switching in a room using Arduino and PIR sensors to prevent any
electricity wastage.
\end{tightemize}

\sectionsep

\section{Education}

\runsubsection{INDIAN INSTITITE OF TECHNOLOGY, KANPUR}
\descript{| B.Tech in Electrical Engineering}
\location{July 2016- April 2020 | Kanpur, Uttar Pradesh (INDIA)}
CPI : 8.1/10
\sectionsep

\runsubsection{D.A.V. PUBLIC SCHOOL, NTS BARKAKANA, C.C.L.}
\descript{| AISSCE}
\location{June 2014 - March 2016 | Ramgarh, Jharkahnd (INDIA)}
Result : 93.4\%
\sectionsep

\runsubsection{D.A.V. PUBLIC SCHOOL, URIMARI}
\descript{| AISSE}
\location{March 2014 | Hazaribagh, Jharkahnd (INDIA)}
CGPA : 10/10
\sectionsep

\section{Skills}
Image Processing \textbullet{} Computer Vision (Basics) \textbullet{} Linux Command Line \textbullet{} Robotics
\section{Languages}
C \textbullet{}   C++ \textbullet{} Python \textbullet{} LATEX \textbullet{} HTML \textbullet{} CSS
\section{Tools}
ROS \textbullet{} OpenCV \textbullet{} Git \textbullet{} SolidWorks \textbullet{} Arduino \textbullet{} Gazebo
\sectionsep
\hfill

\section{Scholastic Achievements}
 Secured rank 3146 at National level in JEE Mains 2016. \\ 
 Secured rank 2477 at National level in JEE Advanced 2016.
\sectionsep

\section{Courses}
Introduction to probability and Statistics (ONGOING) \\
Control Systems (ONGOING) \\
Introduction to Microelectronics (ONGOING) \\
Signal, Systems and Networks \\
Fundamentals of Programming \\
\sectionsep

\section{other campus activities}
July 2017 - Present | Secretary, Robotics Club, IIT Kanpur \\
July 2017 - Present | Secretary, Fine Arts Club, IIT Kanpur \\ 
Feb 2017 | Junior executive, Marketing, Techkriti 2017


\end{document}

