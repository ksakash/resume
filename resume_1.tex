%%%%%%%%%%%%%%%%%%%%%%%%%%%%%%%%%%%%%%%
% This is a modified ONE COLUMN version of
% the following template:
% 
% Deedy - One Page Two Column Resume
% LaTeX Template
% Version 1.1 (30/4/2014)
%
% Original author:
% Debarghya Das (http://debarghyadas.com)
%
% Original repository:
% https://github.com/deedydas/Deedy-Resume
%
% IMPORTANT: THIS TEMPLATE NEEDS TO BE COMPILED WITH XeLaTeX
%
% This template uses several fonts not included with Windows/Linux by
% default. If you get compilation errors saying a font is missing, find the line
% on which the font is used and either change it to a font included with your
% operating system or comment the line out to use the default font.
% 
%%%%%%%%%%%%%%%%%%%%%%%%%%%%%%%%%%%%%%
% 
% TODO:
% 1. Integrate biber/bibtex for article citation under publications.
% 2. Figure out a smoother way for the document to flow onto the next page.
% 3. Add styling information for a "Projects/Hacks" section.
% 4. Add location/address information
% 5. Merge OpenFont and MacFonts as a single sty with options.
% 
%%%%%%%%%%%%%%%%%%%%%%%%%%%%%%%%%%%%%%
%
% CHANGELOG:
% v1.1:
% 1. Fixed several compilation bugs with \renewcommand
% 2. Got Open-source fonts (Windows/Linux support)
% 3. Added Last Updated
% 4. Move Title styling into .sty
% 5. Commented .sty file.
%
%%%%%%%%%%%%%%%%%%%%%%%%%%%%%%%%%%%%%%%
%
% Known Issues:
% 1. Overflows onto second page if any column's contents are more than the
% vertical limit
% 2. Hacky space on the first bullet point on the second column.
%
%%%%%%%%%%%%%%%%%%%%%%%%%%%%%%%%%%%%%%

\documentclass[]{deedy-resume-openfont}


\begin{document}

\lastupdated
\namesection{AKASH}{KUMAR SINGH}{ 
\href{mailto:ksakash@iitk.ac.in}{ksakash@iitk.ac.in} | +91-7318019013 | \href{github.com/ksakash}{Github:ksakash}
}

\section{Work Experience}

\runsubsection{Google Summer of Code}
\descript{| \href{https://robocomp.github.io/web/}{Robocomp}}
\location{May 2018 – Aug 2018 | MENTORS: Marco A Gutiérrez and Ramon Cintas}
\begin{tightemize}
\item The project aimed to integrate \href{https://robocomp.github.io/web/}{Robocomp}, a robotic framework, with a 3D robotic simulator, \href{http://gazebosim.org}{Gazebo}, using \href{https://doc.zeroc.com/ice/3.6/}{zeroc-ice} as a communication middleware.
\item Used Gazebo plugins for robotics interfaces, corresponding to different sensors and actuators, to communicate with the Gazebo simulator.
\item The integration is expected to allow developers more options from the framework and provide a better simulation with a more realistic physics engine.
\end{tightemize}
\sectionsep

\runsubsection{Traffic Light Detection using Deep Leaning}
\descript{| NYU-IITK Research Track}
\location{June 2018 – July 2018 | Prof. Yi Fang (New York University)}
\begin{tightemize}
\item The project aimed to develop a light weight traffic light detection model based on Deep Learning, using various using various model compression techniques.
\item Developed a R-FCN (Region Based Fully Connected Network) model using object-detection tensorflow apis. 
\item Built a feature extractor, using keras, to extract high level features from a trained network, in order to build the meta architecture R-FCN on top of it and retrain it using transfer learning.
\end{tightemize}
\sectionsep

\runsubsection{FRONTEND DEVELOPEMENT}
\descript{| New York Office, IIT Kanpur }
\location{December 2016| Prof. Manindra Agrawal}
\begin{tightemize}
\item Developed new features and improved UI/UX of a scalable web application.
\item Used latest technology stacks like TypeScript in Angular 6 as well as HTML and SCSS for styling while following reactive paradigm using NgRx.
\end{tightemize}
\sectionsep

\section{PROJECTS}
\runsubsection{Autonomous Underwater Vehicle}
\descript{| underwater robotics team, iitk }
\location{February 2017 – Present | Prof. Mangal Kothari}
\begin{tightemize}
\item Developed an \href{https://github.com/ksakash/auv2017-1/blob/IP/task_handler_layer/ip.md}{image processing pipeline}, which can enhance raw underwater images coming from the live video stream through a camera, to get essential information about objects present around the robot. 
\item Developed \href{https://github.com/ksakash/IP_SAUVC/}{Vision Processing ROS package} using OpenCV library and implemented in C++, in order to perform particular tasks in \href{https://sauvc.org/}{SAUVC 2018}.  
\item Worked on Feature Extraction and Matching using various algorithms like SURF, SHIFT, etc. to recognize a known object in a particular scene.
\item Developed a PID based controller for the vehicle to achieve a particular state and configuration for the robot.
\item Developed motion module for the robot, using actionlib provided by ROS, which is responsible for moving the robot on getting a goal from the vision module.
\end{tightemize}
\sectionsep

\runsubsection{TIC-TAC-TOE}
\descript{| Reinforcement Learning }
\location{November 2017 - April 2018| \href{http://home.iitk.ac.in/~nsrivast/}{Prof. Nisheeth Srivastava}}
\begin{tightemize}
\item The project aimed to help an artificial agent learn to play tic-tac-toe game with the help of a  reinforcement learning algorithm called Temporal Difference Learning.
\item Further used the technique of representing the states by set of feature vectors to reduce the state space in order to reduce the time complexity of the algorithm used.
\end{tightemize}
\sectionsep

\runsubsection{CLUB AUTOMATION}
\descript{| Winter Camp }
\location{December 2016| Robotics Club, IITK}
\begin{tightemize}
\item Managed to count the number of people inside a room using PIR sensors so that all electric devices can be turned off in case room is empty.
\item Used Arduino as a microcontroller in order to read data from sensors and perform  calculations.
\end{tightemize}

\sectionsep

\section{Education}

\runsubsection{INDIAN INSTITITE OF TECHNOLOGY, KANPUR}
\descript{| B.Tech in Electrical Engineering}
\location{July 2016- April 2020 | Kanpur, Uttar Pradesh (INDIA)}
CPI : 8.1/10
\sectionsep

\runsubsection{D.A.V. PUBLIC SCHOOL, NTS BARKAKANA, C.C.L.}
\descript{| AISSCE}
\location{June 2014 - March 2016 | Ramgarh, Jharkahnd (INDIA)}
Result : 93.4\%
\sectionsep

\runsubsection{D.A.V. PUBLIC SCHOOL, URIMARI}
\descript{| AISSE}
\location{March 2014 | Hazaribagh, Jharkahnd (INDIA)}
CGPA : 10/10
\sectionsep

\section{Skills}
Image Processing \textbullet{} Computer Vision \textbullet{} Linux Command Line \textbullet{} Robotics \textbullet{} 3D simulation
\section{Languages}
C \textbullet{}   C++ \textbullet{} Python \textbullet{} LATEX \textbullet{} HTML \textbullet{} CSS \textbullet{} Typescript \textbullet{} shell(BASH)
\section{Tools}
ROS \textbullet{} OpenCV \textbullet{} Git \textbullet{} SolidWorks \textbullet{} Arduino \textbullet{} Gazebo \textbullet{} zeroc-ice \textbullet{} Angular \textbullet{} Tensorflow \textbullet{} Keras \textbullet{} GNU octave
\sectionsep
\hfill

\section{Scholastic Achievements}
 Secured rank 3146 at National level in JEE Mains 2016. \\ 
 Secured rank 2477 at National level in JEE Advanced 2016.
\sectionsep

\section{Courses}
Introduction to probability and Statistics \\
Control Systems \\
Introduction to Microelectronics \\
Signal, Systems and Networks \\
Essentials of Scientific Computing \\
Introduction to Electronics \\
Principles of Communication (Under Progress) \\
Data Structures and Algorithms (Under Progress)
\sectionsep

\section{other campus activities}
Secretary, Robotics Club, IIT Kanpur | July 2017 - Mar 2018 \\
Secretary, Fine Arts Club, IIT Kanpur | July 2017 - Mar 2018 \\


\end{document}  \documentclass[]{article}

